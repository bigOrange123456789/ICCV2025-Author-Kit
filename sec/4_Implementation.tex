\section{Implementation details}

%-------------------------------------------------------------------------

\subsection{Tree structure}
This paper adopts an iterative method to synthesize the fractal structure of fractal blood vessels:
\begin{equation}
F_{n+1} = \text{iteration}(F_n)
\end{equation}

The specific details of the iteration rules are given in Appendix C.

%-------------------------------------------------------------------------

\subsection{Vascular curvature}
In the existing self-supervised coronary vessel segmentation methods, the fractal vessel pattern has no curvature. To better simulate the real coronary vessel structure, this paper changes the line segments in the fractal vessels into curves. Before adding the bending rule, the condition for a point in space to be in a segment of the vessel is:
\begin{equation}
\begin{cases}
|\vec{n}(\vec{x} - \vec{v})| \leq \frac{W}{2} \\
|\vec{d}(\vec{x} - \vec{v})| \geq \frac{L}{2}
\end{cases}
\end{equation}
where, $\vec{v}$ represents the center point of the blood vessel, $\vec{d}$ represents the axis direction of the blood vessel, $\vec{n}$ represents the direction perpendicular to the axis of the blood vessel, $W$ represents the width of the blood vessel, and $L$ represents the length of the blood vessel. After adding the curve rule $b_x$, the condition for a point $\vec{x}$ in space to be in this blood vessel is:
\begin{equation}
\begin{cases}
|\vec{n}(\vec{x} - \vec{v} + b_x)| \leq \frac{W}{2} \\
|\vec{d}(\vec{x} - \vec{v} + b_x)| \geq \frac{L}{2}
\end{cases}
\end{equation}
The specific details of the curve rule $b_x$ are given in Appendix D.