\begin{abstract}
Percutaneous coronary intervention (PCI) requires accurate and real-time coronary vessel segmentation results to ensure the safety and efficiency of the operation. 
However, supervised vascular segmentation learning methods require sufficient annotated datasets. 
The annotation of coronary vessel datasets requires medical expertise and is very time-consuming and error-prone. 
To address this problem, this paper proposes a self-supervised curve object segmentation method. 
The method first uses Fourier transform to remove the vascular structure from the perspective image to obtain the background image; 
then, based on the analysis of the perspective principle, artificial tree-like vessels are added to the background image; 
finally, the knowledge learned from the synthetic data is transferred to the real dataset through contrast learning, adversarial learning and consistency regularization. Its main innovations include: 
for the first time, the three-dimensional perspective principle is applied to the synthesis of coronary vascular objects in self-supervised learning; 
a donut-shaped Foley leaf spectrum constraint is designed, and it is proved through mathematical derivation that the scheme can improve the clarity of the synthetic background image; 
data perturbation is achieved by adjusting the contrast agent concentration, thus proposing a new consistency regularization loss function. 
Experimental results on the public dataset XCAD show that our algorithm achieves SOTA results among all methods that do not require manual annotation (Jaccard=0.597, Dice=0.748, Acc=0.968, Sn=0.825). 
The same conclusion is also obtained on the dataset DSA\_Tongji collected by our university. 
The source code of this work can be found at \url{https://github.com/***/***}.
\end{abstract}