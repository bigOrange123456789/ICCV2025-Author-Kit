\section{Related work}
\label{sec:formatting}

%-------------------------------------------------------------------------
\subsection{Synthesis Method of Fractal Images}

Fractal images refer to images whose local shapes are similar to the overall shape. 
Fractal synthesis methods generate geometric patterns with self-similar properties through mathematical iterative formulas, 
providing a parametric data generation paradigm for data-driven models. 
Formula-driven deep learning methods based on fractal theory \cite{01.04.DataBase}\cite{01.05.without_Natural_Images} use synthetic fractal images for pre-training. Its core advantages are: (a) it can generate an infinite amount of training data to achieve sufficient learning; (b) it can measure the difference between images through the difference in fractal parameters, thereby providing more accurate supervision for deep learning. The shape structure of blood vessels is mainly a tree-like bifurcated structure, which is a typical self-similar pattern. In addition, in some local areas of blood vessels, their characteristics such as curvature and direction are similar to the overall vascular network to a certain extent. Therefore, fractal synthesis methods are very suitable for synthesizing vascular patterns \cite{01.02.SSVS}.

%-------------------------------------------------------------------------
\subsection{Segmentation methods that require manual annotation}

Supervised vessel segmentation methods require complex vessel annotations, which are expensive. The U-Net\cite{FC.32.U-Net} architecture has shown good results in the field of medical image segmentation. Octave UNet\cite{SSVS.8} uses Octave Convolution to learn multi-spatial frequency features, thereby achieving the task of segmenting blood vessels in retinal images. Deeper UNet\cite{SSVS.33} replaces the pooling layer of the U-Net encoder with a hierarchical convolution layer, thereby improving the segmentation accuracy of retinal vessels. DLS-MV\cite{SSVS.36} uses the U-Net architecture to achieve major vessel segmentation of X-ray coronary angiography images, and uses other backbone networks such as InceptionResNet-v2 for comparative testing.

Domain adaptation methods solve the medical image segmentation task in the target data domain based on the knowledge in the source data source. In domain adaptation methods, although the target data domain can be unlabeled, the source data domain must be labeled, so it still cannot fundamentally solve the problem of high cost of vascular labeling. MMD \cite{SSVS.3} transferred the knowledge of segmenting mitochondria and synapses in electron microscope images of one brain region to another brain region. YNet\cite{SSVS.31} has performed domain adaptation from brain EM images to HeLa cells and from isotropic FIB/SEM volumes to anisotropic TEM volumes. C-DARL\cite{01.00.C-DARL} proposed contrastive diffusion adversarial representation learning, which achieved vascular segmentation in multiple different data domains such as X-ray coronary angiography and retina using the same network structure. XA-Sim2Real\cite{00.01.XA-Sim2Real} transfers the knowledge of vascular segmentation in the CT data domain to the X-ray data domain, and this method works very well. However, CT is 3D data, which has a larger data volume than 2D data such as X-ray, so XA-Sim2Real requires higher costs for vascular segmentation and annotation.

%-------------------------------------------------------------------------
\subsection{Segmentation methods that do not require manual annotation}

Vessel segmentation architectures using traditional methods rely heavily on predefined mathematical rules. 
This type of architecture uses manually set feature extractors, 
which have limited representation capabilities compared to deep learning methods and therefore perform poorly in experiments. 
YNet\cite{FC.31.YNet} obtains tubular vessel curves through a multi-scale vessel filter based on the Hessian matrix. Hessian\cite{FC.22.Hessian} obtains tubular vessel curves through image gradient flux. Although these methods do not require training, they require manual and fine parameter tuning based on domain knowledge.

Existing unsupervised segmentation methods mainly include clustering-based methods (such as IIC \cite{FC.19.IIC}) and adversarial learning-based methods (such as ReDO \cite{FC.9.ReDO}). Both methods propose segmentation methods based on the characteristics of photographic images, and are therefore not suitable for medical imaging scenarios. IIC assumes that the target has a compact geometric shape (the aspect ratio is close to 1), so it is difficult to capture the tree-like topology of the coronary artery (theoretical aspect ratio is infinite). ReDO relies on significant visual contrast between the target and the background. Since X-ray angiography is a perspective imaging, the pixels in the vascular area will be affected by the background tissue at the bottom of the blood vessel, which makes it difficult for ReDO to accurately distinguish the features between the target and the background.

Self-supervised learning trains models by generating labels from the data itself without the need for manual data annotation, thus solving the problem of high labeling costs for vascular image segmentation. Self-supervised learning has achieved excellent performance in classification\cite{FC.26}\cite{FC.4}, detection\cite{FC.42}\cite{FC.5}, and image translation\cite{FC.29}\cite{FC.41}. SSVS\cite{01.02.SSVS} proposed a self-supervised vascular segmentation method based on fractal synthesis algorithm and adversarial learning, which achieved excellent results on a variety of different categories of datasets such as X-ray coronary angiography and retina. DARL\cite{01.01.DARL} achieved better results than SSVS through diffusion adversarial representation learning. Both SSVS and DARL require background image data as auxiliary data, while FreeCOS\cite{02.01.FreeCOS} proposed a background image synthesis method based on Fourier transform, so that self-supervised learning can be completed even without real background image data.

Our method is a superior alternative to the previous self-supervised learning blood vessel segmentation methods. 
Our method synthesizes fractal blood vessels (As shown in Figure \cref{fig:my01}) based on X-ray perspective analysis, 
and achieves better results than the previous self-supervised learning blood vessel segmentation methods. 
In addition, this paper also addresses two major challenges in XCA segmentation:
\begin{quotation}
  \noindent
The challenge of domain shift: The concentration of contrast agents in the blood can significantly affect imaging results, resulting in domain shift. In order to enable the neural network to learn the knowledge that the concentration of contrast agent does not affect the geometric structure of blood vessels, this paper designs a data perturbation algorithm based on contrast agent concentration transformation.

The challenge of information coupling: Due to the use of perspective imaging in XCA technology, the vascular area on the contrast image also contains a large amount of background information, resulting in severe coupling between the two. In order to address the challenge of information coupling, this paper proposes a new spectrum segmentation method (double ellipse constraint).
  \end{quotation}
  
%-------------------------------------------------------------------------

\begin{figure}[htbp]
  \centering
  \begin{subfigure}[t]{0.2\textwidth}
      \centering
      \includegraphics[width=\linewidth]{pic/1.a.png}
      \caption{Synthetic vessels of SSVS, DARL, and FreeCOS}
      \label{fig:my01a}
  \end{subfigure}
  \hspace{0.005\textwidth} % 调整间距为页面宽度的2%
  \begin{subfigure}[t]{0.2\textwidth}
      \centering
      \includegraphics[width=\linewidth]{pic/1.b.png}
      \caption{Synthetic vessels in this paper}
      \label{fig:my01b}
  \end{subfigure}
  \caption{Comparison of fractal synthesis of blood vessels between this paper and other methods} % 为整个figure环境设置大标题
  \label{fig:my01}
\end{figure}
